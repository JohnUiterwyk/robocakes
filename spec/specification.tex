\documentclass[a4paper]{article}

\usepackage{hyperref}

\begin{document}

\title{Robocakes Project Specification}
\author{John Uiterwky, Virginia King, Stewart Johnston}

\maketitle

\section{Introduction}

Our project is to design and implement a small multiplayer game, played over
a local area network.

The preferred final outcome is a game that is played over several monitors,
each driven by a pi. The monitors will be placed side-by-side and the
generated map will span across all the screens.

\section{System Overview}

The project will consist of several parts, with coupling kept to a minimum to
allow iterative improvements depending on the team's progress. A networking
component will allow the systems to communicate with one another, and all the
systems will manage their own graphical output.

\section{Design Considerations}

The project will use a modified MVC pattern and attempt to minimise coupling,
especially with regards to the user interface. Later milestones involve
experimenting with different UI hardware, so an ability to quickly plug in
different UI modules will be very helpful.

\subsection{Goals and Objectives}

The project will split goals up into milestones. This will allow the project
to satisfy a broad range of learning objectives as soon as possible, and give
us time to add the more interesting features at a later stage.

\begin{enumerate}
    %TODO More goals
    %TODO Add milestones
    %TODO
  \item Packaging project as a deb, rpm, pacman package?
\end{enumerate}

\subsection{Assumptions and Dependencies}

Our focus will be on developing the network communication, and as such we will
implement a custom networking interface using linux sockets.

With that in mind, we aim to accelerate progress elsewhere as much as possible
by using existing libraries in other areas of the project.\footnote{These
dependencies may change.}

\begin{itemize}
  \item Physics library will be provided by
    \href{http://chipmunk2d.net/}{Chipmunk2d}
  \item Graphics library will be provided by
    \href{http://www.khronos.org/openvg/}{OpenVG}
\end{itemize}

\subsection{General Constraints}

The position of the monitor in relation to other monitors will be very
difficult to determine programmatically. Initially, screen positions,
orientations and dimensions will be included as a configuration file on each
device. This will save a lot of time, since interrogating monitors to
determine supported resolutions and dimensions is hard.

\section{Architecture}

\subsection{Development Methodology}

The project will use an agile development methodology. We will use
\href{https://www.toggl.com/}{Toggl} for time tracking, and 
\href{https://trello.com/}{Trello} for task tracking.

\subsubsection{Sprint Schedule}

\begin{enumerate}
  \item Network communication over UDP

    Unidirectional communication over UDP for server broadcasts to
    clients.

  \item Network communication over TCP

    Bidirectional communication between client and server.

  \item foo %TODO ... ???
  \item User input using ultrasonic distance sensors.
\end{enumerate}

The project will use a single server to provide overall control, and clients
will connect to the server to join a game.

The server may run a client locally as well, depending on performance
constraints.

The networking component will use UDP to communicate, but a future milestone
includes migrating to TCP/IP and comparing the protocols.

Each client will manage it's own display; retrieving map information from the
server and rendering locally. Updated map information will be distributed to
clients by the server as needed.

\subsection{System Design}

To achieve maximum performance, a stripped down linux installation running
a minimum of services will be used for demonstration purposes. However, the
project aims to develop at least a deb of the project for easy installation on
debian-based distributions. Other packages will be investigated as time
permits.

\subsection{Data Design}

The network component will use JSON to distribute information to the clients.
%TODO More data design

\subsection{Program Design}
%TODO Program Design

\subsubsection{Detailed Module Design}

\begin{enumerate}
    %TODO All this stuff
  \item Model
  \item View
  \item Controller
  \item Network

\end{enumerate}

\section{Testing Issues}
%TODO Issues?

\subsection{Types of Testing to be conducted}

%TODO More tests
% Gin says that these tests will form the basis against which we will be
% assessed.
\begin{itemize}
  \item Base-level libraries
    \begin{itemize}
      \item uClibC vs glibc
      \item Does busybox provide performance improvement over separate tools?
    \end{itemize}
  \item Network
    \begin{itemize}
      \item Comparison between udp and tcp
      \item Compression (cpu intensive) vs uncompressed (network intensive)
        data.
    \end{itemize}
\end{itemize}

\subsection{Performance Bounds}
%TODO Performance bounds?

\section{Roles and Responsibilities}

\begin{itemize}
  \item John Uiterwyk - developer

    John is our chief developer and agile expert. He will lead the development
    and also provide Gin with agile project management advice if needed.

  \item Virginia King (Gin) - documenter/manager

    Gin will serve as a project manager and work on collating the project's
    documentation. She will also be responsible for the graphics development.

  \item Stewart Johnston (stooj) - linux

    stooj will focus on platform requirements, building a stock linux build
    for the team to work from, and working to improve the deployment and
    testing workflow.

\end{itemize}

\section{Biography}
%TODO Does this mean bibliography?

\end{document}
